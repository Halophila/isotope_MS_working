%%%%%%%%%%%%%%%%%%%%%%% file template.tex %%%%%%%%%%%%%%%%%%%%%%%%%
%
% This is a general template file for the LaTeX package SVJour3
% for Springer journals.          Springer Heidelberg 2010/09/16
%
% Copy it to a new file with a new name and use it as the basis
% for your article. Delete % signs as needed.
%
% This template includes a few options for different layouts and
% content for various journals. Please consult a previous issue of
% your journal as needed.
%
%%%%%%%%%%%%%%%%%%%%%%%%%%%%%%%%%%%%%%%%%%%%%%%%%%%%%%%%%%%%%%%%%%%
%
% First comes an example EPS file -- just ignore it and
% proceed on the \documentclass line
% your LaTeX will extract the file if required
%\begin{filecontents*}{example.eps}
%!PS-Adobe-3.0 EPSF-3.0
%%BoundingBox: 19 19 221 221
%%CreationDate: M
%\documentclass{svjour3}                     % onecolumn (standard format)
\documentclass{report}     % onecolumn (ditto)
%\documentclass[smallcondensed]{svjour3}       % onecolumn (second format)
%\documentclass[twocolumn]{svjour3}          % twocolumn
%
%\smartqed  % flush right qed marks, e.g. at end of proof
\usepackage[letterpaper, margin=0.9in]{geometry}
\usepackage{graphicx}
\usepackage{multirow}
\usepackage[authoryear,round]{natbib}
\usepackage{caption}
\usepackage{booktabs}
%\usepackage[sectionbib]{chapterbib}
\usepackage[usenames, dvipsnames]{color}   %May be necessary if you want to color links
\usepackage[hidelinks,linktoc=all]{hyperref}
\usepackage{textgreek}
\usepackage{mathpazo}
\usepackage{newtxtext,newtxmath,amsmath}
\usepackage[utf8]{inputenc}
\renewcommand{\arraystretch}{1.5}
\usepackage[
singlelinecheck=false 
]{caption}
\pagenumbering{gobble}
\usepackage{textcomp}


\makeatletter
\renewcommand*{\thetable}{\arabic{table}}
\renewcommand*{\thefigure}{\arabic{figure}}
\let\c@table\c@figure
\makeatother 

%\renewcommand{\figurename}{Online Resource}
%\usepackage[figurename=Online Resource]{caption}
\captionsetup[figure]{labelfont={bf},name={Online Resource},labelsep=space}
\captionsetup[table]{labelfont={bf},name={Online Resource},labelsep=space}
%\usepackage[labelfont=bf]{caption}
%
% \usepackage{mathptmx}      % use Times fonts if available on your TeX system
%
% insert here the call for the packages your document requires
%\usepackage{latexsym}
% etc.
%
% please place your own definitions here and don't use \def but
% \newcommand{}{}
%
% Insert the name of "your journal" with
% \journalname{myjournal}
%
\begin{document}








\section*{Spatial variation in organic matter recalcitrance and $\delta$\textsuperscript{13}C of organic and inorganic carbon in seagrass sediments as indicators of carbon cycling and exchange processes}


\subsection*{Supplementary material \bigskip}
Jason L. Howard $\cdot$ Vicki Absten $\cdot$ Christian C. Lopes $\cdot$ Sara S. Wilson $\cdot$ James W. Fourqurean



\bigskip
\bigskip

\begin{figure}[h]
  \centering
   \includegraphics[width=.99\textwidth,clip, trim={0.1mm 0.1mm 0.1mm 0.1mm}]{map_Corg_sup.png}
\caption{C\textsubscript{org} content in surface soils across South Florida sampling sites.}
  \label{Online Resource:2F0}
\end{figure}


\begin{figure}[h]
  \centering
   \includegraphics[width=.99\textwidth,clip, trim={0.1mm 0.1mm 0.1mm 0.1mm}]{ramped_loi}
\caption{Weight loss during controlled oxidation from from 65 \textdegree C to 600 \textdegree C. Top panel shows decrease in sample weight as temperature increases, while bottom panel shows percent of total weight loss at each oxidation step. Data represent mean $\pm$ SE (n = 90).}
  \label{Online Resource:2F1}
\end{figure}



\begin{figure}[h]
  \centering
  \includegraphics[width=.95\textwidth]{sup_bigplot}
\caption{Relationship between LOI and decrease in C content for each temperature step. Lines represent significant correlations (linear regression, p < 0.05).}
  \label{fig:2F2}
\end{figure}

\clearpage


\end{document}




% end of file template.tex